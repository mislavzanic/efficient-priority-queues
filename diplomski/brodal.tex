\newenvironment{myindentpar}[1]%
 {\begin{list}{}%
         {\setlength{\leftmargin}{#1}}%
         \item[]%
 }
 {\end{list}}

\chapter{Brodalov Queue}

\section{Uvod}
Apstraktna struktura podataka prezentirana u ovom poglavlju zadovoljava asimptotske granice $O(1)$ za \texttt{FindMin}, \texttt{Meld}, \texttt{Insert} i \texttt{DecreaseKey}, te $O(\log n)$ za \texttt{DeleteMin}.
U poglavlju 3.3 opisujem pomo\'{c}nu strukturu podataka potrebnu za realizaciju nekih ograni\v{c}enja Brodalovog reda.

\section{Opis ASP-a}
U ovom potpoglavlju navodim pravila koja Brodalov red treba po\v{s}tovati izmedu svake dvije operacije, dokazujem tvrdnje koje proizlaze iz tih pravila i obja\v{s}njavam okvirnu strukturu Brodalovog reda.
Prioritetan red definiramo kao skup od dva stabla $T_{1}$ i $T_{2}$.
\v{C}vorove korijene stabla $T_{i}$ ozna\v{c}avati \'{c}u sa $t_{i}$.
Ideja iza 2 stabla je ta da $t_{1}$ bude najmanji element
Dalje navodim invarijante koje struktura podataka treba zadovoljavati.

\begin{defn}[Invarijante \v{c}vorova]\label{S_inv}
  Neka je $x$ \v{c}vor Brodalovog reda. Tada za \v{c}vor x trebaju vrijediti sljede\'{c}e invarijante:
  \begin{myindentpar}{15pt}
    S1: Ako je $x$ list, onda je $r(x) = 0$, \\
    S2: $r(x) < r(p(x))$, \\
    S3: Ako $r(x) > 0$, onda $n_{r(x)-1}(x) \ge 2$, \\
    S4: $n_{i}(x) \in \{0,2,\ldots,7\}$, \\
    S5: $T_{2} = \emptyset $ \bigskip ili \bigskip $r(t_{1}) \le r(t_{2})$.
  \end{myindentpar}
\end{defn}

S1 i S2 ka\v{z}u da je rang lista jednak 0 i da rang \v{c}vora $x$ strogo raste prema korijenu.
S3 ka\v{z}e da \v{c}vor ranga $k > 0$ mora imat barem dvoje dijece ranga $k - 1$.
S5 ka\v{z}e da je $T_{2}$ prazno stablo, ili ima rang ve\v{c}i od $T_{1}$
S4 ka\v{z}e da je broj djece ranga $i$ \v{c}vora $x$ konstantan i ne smije biti 1.
Ne dopu\v{s}tamo da broj \v{c}vorova stabla bude 1 kako bi i dalje vrijedila invarijanta S3 nakon \v{s}to odre\v{z}emo dijecu najve\v{c}eg ranga nekog \v{c}vora. $n_{i}(x) \le 7$ je posljedica konstrukcije koju kasnije obja\v{s}njavamo.
Iz S1 i S3 slijedi sljede\'{c}a lema.

\begin{lem}\label{Exp_cv}
  Neka je $x$ \v{c}vor nekog stabla $T$ koje zadovoljava S1 i S3. Podstablo sa korijenom $x$, ranga $r(x)$ ima barem $2^{r(x)+1} - 1$ \v{c}vorova.
\end{lem}
\begin{proof}
  Dokaz indukcijom po rangu \v{c}vora $x$.\\
  \emph{Baza}: $r(x) = 0$.\\
  \indent Iz S1 slijedi da je $x$ list, pa podstablo sa korijenom u $x$ ima $1 = 2^{0 + 1} - 1$ \v{c}vorova.\\
  \emph{Pretpostavka}: za \v{c}vor $x$ ranga $r(x) = n \in \N$ vrijedi tvrdnja.\\
  \emph{Korak}:\\
  \indent Neka je $x$ \v{c}vor ranga $r(x) = n+1$. Tada, po S3, slijedi da $x$ ima barem dvoje dijece ranga $n$. Tada podstablo sa korijenom u \v{c}voru $x$ ima barem $2(2^{k + 1} - 1)$ dijece iz \v{c}ega slijedi tvrdnja.
\end{proof}

\begin{lem}\label{Log_cv}
  Neka je $x$ \v{c}vor ranga $r(x)$ nekog stabla $T$ koje zadovoljava S1-S5 i neka je $n \in \N$ broj \v{c}vorova u $T$. Tada su rang i stupanj \v{c}vora $x$ $O(\log n)$.
\end{lem}
\begin{proof}
  Dovoljno je tvrdnju pokazat za korijen stabla $T$. Ozna\v{c}imo ga sa $t$.
  Iz S1-S4 slijedi da $n$ mo\v{z}emo ograni\'{c}iti odozgo sa $7^{r(t)} + 1$ i da se ta granica mo\v{z}e posti\'{c}i (svi \v{c}vorovi osim listova imaju po 7 djece svakog ranga manjeg od svog ranga).
  Iz toga slijedi $r(t) \in O(\log n)$, a iz toga i da je stupanj $t$ iz $O(\log n)$.
\end{proof}



\begin{defn}[Invarijante strukture]\label{O_inv}
  Neka je $w_{i}(x)$ broj \v{c}vorova u $W(x)$ ranga $i$.
  \begin{myindentpar}{15pt}
    O1: $t_{1} = \min T_{1} \cup T_{2}$, \\
    O2: ako $y \in V(x) \cup W(x)$, onda $y \ge x$, \\
    O3: ako $y \le p(y)$, onda $\exists x \ne y$ tako da $y \in W(x) \cup V(x)$, \\
    O4: $w_{i}(x) \le 6$, \\
    O5: ako $V(x) = (y_{|V(x)| - 1},\dotsc,y_{1},y_{0})$, onda $r(y_{i}) \ge \lfloor\frac{i}{\alpha}\rfloor$, za $i = 0,1,\dotsc,|V(x)| - 1$, gdje je $\alpha$ konstanta.
  \end{myindentpar}
\end{defn}

\begin{defn}[Invarijante korijena]\label{R_inv}
  Neka je x \v{c}vor Brodalovog reda. Tada za \v{c}vor x trebaju vrijediti sljede\'{c}e invarijante:
  \begin{myindentpar}{15pt}
    R1: $n_{i}(t_{j}) \in \{2,3,\dotsc,7\}$, za $i = 0,1,\dotsc,r(t_{j}) - 1$,\\
    R2: $|V(t_{1})| \le \alpha r(t_{1})$, \\
    R3: ako $y \in W(t_{1})$, onda $r(y) < r(t_{1})$.
  \end{myindentpar}
\end{defn}

\section{Vodilja}
U ovom dijelu opisujem pomo\v{c}nu apstraktnu strukturu podataka \emph{vodilji}. Probleme koje rije\v{s}avaju vodilje mo\v{z}e se opisati ovako.

\begin{defn}[Problem vodilje]
  Pretpostavimo da imamo niz cijelih brojeva $x_{n},\ldots,x_{1}$ za koji \v{z}elimo da vrijedi invarijanta $x_{i} \le T, T \in \Z, i \in \{1,\ldots,n\}$.
  Za svaki inkrement nekog od $x_{i}, i \in \{1,\ldots,n\}$ t.d. $x_{i} > T$ trebamo $O(1)$ operacija redukcije kojima ponovno uspostavljamo invarijantu.
\end{defn}

\begin{defn}
  Neka je $x_{n},\ldots,x_{1}$ proizvoljan niz cijelih brojeva.
  Operacija redukcije nad $x_{i}, i \in \{1,\ldots,n-1\}$ definiramo kao dekrement $x_{i}$ za barem 2 i inkrement $x_{i+1}$ za najvi\v{s}e 1.
  Za $x_{n}$, operaciju redukcije definiramo analogno, bez inkrementa sljede\v{c}eg elementa niza.
\end{defn}

\section{Opis implementacije}
